\section{Derivation of discretized free energy}

The dimensionless Ginzburg-Landau (GL) free energy density for a two-component $p_x\pm ip_y$ order parameter used in \cite{AsleGaraud16} is given by
\begin{equation}
  \begin{split}
	\mathcal{F} = &|\v{\nabla}\times\v{A}|^2 + |\v{D}\eta^+|^2 + |\v{D}\eta^-|^2\\
	&+ (\nu+1)\Re\big[(D_x\eta^+)^\ast D_x\eta^- - (D_y\eta^+)^\ast D_y\eta^-\big]\\
	& + (\nu-1)\Im\big[(D_x\eta^+)^\ast D_y\eta^- + (D_y\eta^+)^\ast D_x\eta^-\big]\\
	&+ 2|\eta^+\eta^-|^2 + \nu\Re\big( (\eta^+)^{\ast 2}(\eta^-)^2 \big) + \sum_{h=\pm}\Big[-|\eta^h|^2 + \frac{1}{2}|\eta^h|^4\Big],
  \end{split}
  \label{eq:disc:FreeEn:initial}
\end{equation}
where $\eta^\pm=\rho_\pm e^{i\theta^\pm}$ are the components of the superconducting order parameter. The lengths are given in terms of $\xi = [\alpha_0(T-T_c)]^{-1/2}$,
the magnetic field $\v{B} = \v{\nabla}\times\v{A}$ is given in units of $\sqrt{2}B_C = \Phi_0/(2\pi\lambda\xi)$. The dimensionless gauge
coupling used in the covariant derivatives $\v{D} = \v{\nabla} + ig\v{A}$ is used to parametrize the ratio of two length scales 
$g^{-1} = \kappa = \lambda/\xi$. On the first line of the equation, we see the Maxwell term as well as the normal kinetic terms associated with the order parameter components. The second
line exhibits Andreev-Bashkin terms while the third line contains what we call mixed gradient terms (MGT) where both different components and gradients are mixed.

Note that the theory is gauge invariant under the local gauge transformation
\begin{subequations}
  \begin{align}
	\eta^h &\mapsto e^{i\lambda}\eta^h \label{eq:disc:Gauge:continuous:eta}\\
	\v{A} &\mapsto \v{A} - \frac{1}{g}\v{\nabla}\lambda.
	\label{eq:disc:Gauge:continuous:A}
  \end{align}
  \label{eq:Gauge:continuous}
\end{subequations}

\subsection{London approximation}

Assuming that the amplitude of the GL order parameters are constant (London approximation),
such that $\eta^h(\v{r}) = \rho_he^{i\theta^h(\v{r})}$ and inserting that the components of the
covariant derivative is given by $D_\mu = \partial_\mu + igA_\mu$, the free energy based on the density $\mathcal{F}$ in Eq.~\eqref{eq:disc:FreeEn:initial} becomes
\begin{equation}
  \begin{split}
	&F^\text{lon} = \int\!\!\mathrm{d}^2r\,\Big\{|\v{\nabla}\times\v{A}|^2 + \sum_{\mu,h}\rho_h^2\big(\partial_\mu\theta^h + gA_\mu\big)^2\\
	&+\rho_+\rho_-(\nu+1)\cos(\theta^+-\theta^-)\Big[\big(\partial_x\theta^+ + gA_x\big)\big(\partial_x\theta^- + gA_x\big) - \big(\partial_y\theta^+ + gA_y\big)\big(\partial_y\theta^- + gA_y\big)\Big]\\
	&+\rho_+\rho_-(\nu-1)\sin(\theta^--\theta^+)\Big[\big(\partial_x\theta^- + gA_x\big)\big(\partial_y\theta^+ + gA_y\big) + \big(\partial_y\theta^+ + gA_y\big)\big(\partial_x\theta^+ + gA_x\big)\Big]\\
  & + \nu\rho_+^2\rho_-^2\cos 2(\theta^+-\theta^-)\Big\} + \mathcal{V}\Big\{\sum_h\Big[-\rho_h^2 + \frac{1}{2}\rho_h^4\Big] + 2\rho_+\rho_-\Big\}.
  \end{split}
  \label{eq:disc:FreeEn:London}
\end{equation}
In the sums in this equation, $h\in\{\pm\}$ while $\mu\in\{x,y\}$. $\mathcal{V}$ denotes the volume of the system. The terms proportional to $\mathcal{V}$ are usually ignored, but since
we will consider globally varying $\rho_\pm$ in the Monte-Carlo simulation, we have included them here for completness.

\subsection{Lattice regularization}

The discretization of the continuum models above onto a two-dimensional lattice is done through a number of mappings \cite{troels14,peder16,shimizu12}. Integrals over space are mapped to a sum over lattice positions $\v{r}$ by
\begin{equation}
  \int\!\!\mathrm{d}^Dx\, \mapsto a^D\sum_\v{r},\\
  \label{eq:disc:map:int}
\end{equation}
where $a$ is the length between lattice sites. Since the numerical lattice has square symmetry, only one length parameter is necessary. The $6$ degrees of freedom go from being continuous variables
$\eta^\pm,\,\eta^{\pm\;\ast}, \v{A}$, to begin discretized variables where the superconducting order parameter gets an associated variable at each lattice site $\eta^\pm_{\v{r}}, \eta^{\pm\;\ast}_\v{r}$,
while the vector potential $\v{A}$ is discretized by link variables $A_{\v{r},\mu}$ between the lattice site at $\v{r}$ and the nearest neighbor in direction $\hat{\mu}$. Covariant derivatives are replaced by
\begin{equation}
  D_\mu\eta^h = (\partial_\mu + igA_\mu)\eta^h \mapsto a^{-1}\big(\eta^h_{\v{r}+\hat{\mu}}e^{igaA_{\v{r},\mu}}-\eta^h_\v{r}\big).
  \label{eq:disc:map:covarDer}
\end{equation}
Note the abuse of vector notation in $\eta^h_{\v{r}+\hat{\mu}}$ which is a shorthand for the nearest neighbor of the lattice site $\v{r}$ in the $\hat{\mu}$ direction. In this notation, the definition
of the difference operator $\Delta_\mu$ becomes
\begin{equation}
  \Delta_\mu A_{\v{r},\nu} = A_{\v{r}+\hat{\mu},\nu} - A_{\v{r},\nu}.
  \label{eq:disc:difference}
\end{equation}
Finally the continuous version of the Maxwell term which is responsible for the free energy of the electromagnetic field, is replaced by a sum over all plaquettes of the lattice
\begin{equation}
  \begin{split}
	\int\!\!\mathrm{d}^Dr\,(\v{\nabla}\times\v{A})^2 \mapsto a^{D-2}\sum_\v{r}(\v{\Delta}\times\v{A}_\v{r})^2 &= a^{D-2}\sum_{\v{r},\mu}(\epsilon_{\mu\nu\lambda}\Delta_\nu A_{\v{r},\lambda})^2 \\
	&= a^{D-2}\sum_{\v{r},\mu>\nu}(\Delta_\mu A_{\v{r},\nu} - \Delta_\nu A_{\v{r},\mu})^2.
  \end{split}
  \label{eq:disc:Maxwell:noncompact}
\end{equation}
This is the noncompact way of discretizing the vector potential, i.e. $A_{\v{r},\mu}\in(-\infty,\infty)$. In the compact version, the link variables $A_{\v{r},\mu}$ have $2\pi$ periodicity and generally
allows topologically nontrivial excitations such as magnetic monopoles \cite{shimizu12} for large fluctuations of the vector potential.

To ensure proper boundary conditions of the vector potential we will split it in a constant and fluctuating field such that $\v{A}_\v{r} = \v{A}_\v{r}^0 + \v{A}_\v{r}^f$, where 
$\v{A}^0_\v{r} = (0, 2\pi f r_x, 0)^\mathrm{T}$. Here $f$ is the magnetic filling fraction while $r_x$ is the $x$ position of lattice site $\v{r}$. Letting the fluctuating field $A_\v{r}^f$ have
periodic boundary conditions hence ensures that there will be a constant external magnetic field $\v{B} = (0,0, 2\pi f)$ penetrating the lattice \cite{Smorgrav05}. The Maxwell term then is divided in
three terms
\begin{equation}
  F_A = a^{D-2}\sum_\v{r}\Big[(\v{\Delta}\times\v{A}_\v{r}^0)^2 + (\v{\Delta}\times\v{A}_\v{r}^f)^2 + 2(\v{\Delta}\times\v{A}_\v{r}^0)\cdot(\v{\Delta}\times\v{A}_\v{r}^f)\Big],
  \label{eq:disc:Maxwell:noncompact:split}
\end{equation}
where the first one is constant and hence it can be neglected. Since the model under concideration is of a $2D$ lattice, the link variables $A_{\v{r},z}$ out of the plane are neglected as well as any
difference in the $z$-direction $\Delta_zA_{\v{r},\mu}$. This simplifies the expression for $\v{\Delta}\times\v{A}_\v{r}$ in Eq.~\eqref{eq:disc:Maxwell:noncompact} to only the $z$-component
$\v{\Delta}\times\v{A}_\v{r} = \hat{z}\big(\Delta_xA_{\v{r},y} - \Delta_yA_{\v{r},x}\big)$. Denoting the $z$-component of the discretized curl of the fluctuating field
\begin{equation}
  A^f_{\Box,\v{r}} \equiv \Delta_xA^f_{\v{r},y} - \Delta_yA^f_{\v{r},x} = A^f_{\v{r},x} + A^f_{\v{r}+\hat{x},y} - A^f_{\v{r}+\hat{y},x} - A^f_{\v{r},y},
  \label{eq:disc:plaquetteSum}
\end{equation}
this can be interpreted as the sum of the fluctuating vector potential over the plaquette $\Box$ at the lattice site $\v{r}$. With this definition, the Maxwell term 
in Eq.~\eqref{eq:disc:Maxwell:noncompact:split} can be written
\begin{equation}
  \begin{split}
	F_A &= \sum_\v{r}\Big\{\big(A^f_{\Box,\v{r}}\big)^2 + 2\big(\Delta_xA^0_{\v{r},y} - \Delta_yA^0_{\v{r},x}\big)A^f_{\Box,\v{r}}\Big\}\\
	&= \sum_\v{r}\big(A^f_{\Box,\v{r}}\big)^2 + 2\cdot2\pi fa\sum_{\v{r}}A^f_{\Box,\v{r}} = \sum_\v{r}\big(A^f_{\Box,\v{r}}\big)^2.
  \end{split}
  \label{eq:disc:Maxwell:plaquetteSum}
\end{equation}
In the second line we have evaluated the specific expression for the constant vector potential. In the third, we realize that the sum over $A^f_{\Box,\v{r}}$ vanishes
by shifting lattice indices.

Applying this procedure to the free energy density in Eq.~\eqref{eq:disc:FreeEn:initial} yields the free energy $F$, which we divide into terms
\begin{equation}
  F^\text{disc} = F_K + F_\text{MGT} + F_V + F_A.
  \label{eq:disc:FreeEn:disc}
\end{equation}
The potential term is given by
\begin{equation}
  F_V = a^2\sum_\v{r}\Big\{\sum_h\big[-|\eta^h_\v{r}|^2 + \frac{1}{2}|\eta^h_\v{r}|^4\big] + 2|\eta^+_\v{r}\eta^-_\v{r}|^2 + \nu\Re\big[(\eta^+_\v{r})^{\ast\,2}(\eta_\v{r}^-)^2\big]\Big\},
  \label{eq:disc:FreeEn:disc:potential}
\end{equation}
while the Maxwell term $F_A$ is given by the expression to the right in Eq.~\eqref{eq:disc:Maxwell:plaquetteSum}. The variables on the numerical lattice is assumed to be periodic - 
when e.g. $\eta_{\v{r}+\hat{\mu}}^h$ is evaluated at a lattice site $\v{r} + \hat{\mu}$ outside the lattice, we wrap around. Utilizing this assumption, the regularized representation of the normal
kinetic terms without mixed gradients becomes
\begin{equation}
  F_K = \sum_{\v{r},\mu,h}\Big\{2|\eta^h_\v{r}|^2 - \big[(\eta^h_{\v{r}+\hat{\mu}})^\ast\eta^h_\v{r}e^{-igaA_{\v{r},\mu}} + \text{c.c.}\big]\Big\}.
  \label{eq:disc:FreeEn:disc:kin}
\end{equation}
The mixed gradient terms of $\mathcal{F}$ in Eq.~\eqref{eq:disc:FreeEn:initial} forms the regularized free energy
\begin{equation}
  \begin{split}
	F_\text{MGT} = \sum_\v{r}\Big\{&(\nu+1)\Re\Big[\eta^{-\;\ast}_\v{r}\big(\eta^+_{\v{r}+\hat{y}}e^{igaA_{\v{r},y}}-\eta^+_{\v{r}+\hat{x}}e^{igaA_{\v{r},x}}\big)
	+ \eta^+_\v{r}\big(\eta^{-\;\ast}_{\v{r}+\hat{y}}e^{-igaA_{\v{r},y}} - \eta^{-\;\ast}_{\v{r}+\hat{x}}e^{-igaA_{\v{r},x}}\big)\Big]\\
	+&(\nu-1)\Im\Big[\big(\eta^{+\;\ast}_{\v{r}+\hat{y}}\eta^-_{\v{r}+\hat{x}} - \eta^{-\;\ast}_{\v{r}+\hat{y}}\eta^+_{\v{r}+\hat{x}}\big)e^{iga(A_{\v{r},x}-A_{\v{r},y})} + 2\eta^{+\;\ast}_\v{r}\eta^-_\v{r}\\
	  &\qquad\quad\;\;\;+\big(\eta^{-\;\ast}_\v{r}\eta^+_{\v{r}+\hat{x}} - \eta^{+\;\ast}_\v{r}\eta^-_{\v{r}+\hat{x}}\big)e^{igaA_{\v{r},x}}
	+ \big(\eta^{-\;\ast}_{\v{r}+\hat{y}}\eta^+_\v{r} - \eta^{+\;\ast}_{\v{r}+\hat{y}}\eta^-_\v{r}\big)e^{-igaA_{\v{r},y}}\Big]\Big\}.
  \end{split}
  \label{eq:disc:FreeEn:disc:MGT}
\end{equation}

The regularized free energy is invariant under the gauge-transformation
\begin{subequations}
  \begin{align}
	\eta^h_\v{r}&\mapsto e^{i\lambda_\v{r}}\eta^h_\v{r} \label{eq:disc:Gauge:disc:A}\\
	A_{\v{r},\mu} &\mapsto A_{\v{r},\mu} - \frac{\Delta_\mu\lambda_\v{r}}{ga}.
	\label{eq:disc:Gauge:disc:A}
  \end{align}
  \label{eq:disc:Gauge:disc}
\end{subequations}

\subsection{Lattice regularization + London approximation}

Similarly to what we did for Eq.~\eqref{eq:disc:FreeEn:London}, we assume that the amplitude of the components of the order-parameter are constant such that $\eta^h_\v{r} = \rho_he^{i\theta^h_\v{r}}$.
Inserting this assumption into the different terms of the free energy $F$ in Eq.~\eqref{eq:disc:FreeEn:disc} yields the normal kinetic free energy
\begin{equation}
  F^\text{lon}_K = 2\sum_{\v{r},\mu,h}\rho_h^2\Big\{1-\cos\big(\Delta_\mu\theta^h_\v{r}+gaA_{\v{r},\mu}\big)\Big\},
  \label{eq:disc:FreeEn:lonDis:K}
\end{equation}
the free energy given by the potential terms
\begin{equation}
  F^\text{lon}_V = a^2\mathcal{N}\Big\{\sum_h\big[-\rho_h^2 + \frac{1}{2}\rho_h^4\big] + (2-\nu)\rho_+^2\rho_-^2\Big\} + 2\rho_+^2\rho_-^2a^2\nu\sum_\v{r}\cos^2\big(\theta^+_\v{r}-\theta^-_\v{r}\big),
  \label{eq:disc:freeEn:lonDis:V}
\end{equation}
as well as the MGT free energy
\begin{equation}
  \begin{split}
	&F^\text{lon}_\text{MGT} = \rho_+\rho_-\sum_\v{r}\Big\{(\nu+1)\Big[\cos\big(\theta^+_{\v{r}+\hat{y}} - \theta^-_\v{r} + gaA_{\v{r},y}\big) - \cos\big(\theta^+_{\v{r}+\hat{x}}-\theta^-_\v{r} + gaA_{\v{r},x}\big)\\
	&\qquad\qquad+ \cos\big(\theta^-_{\v{r}+\hat{y}} - \theta^+_\v{r} + gaA_{\v{r},y}\big) - \cos\big(\theta^-_{\v{r}+\hat{x}} - \theta^+_\v{r} + gaA_{\v{r},x}\big)\Big]\\
	&+(\nu-1)\Big[\sin\big(\theta^-_{\v{r}+\hat{x}} - \theta^+_{\v{r}+\hat{y}} + ga(A_{\v{r},x} - A_{\v{r},y})\big) + \sin\big(\theta^-_{\v{r}+\hat{y}} - \theta^+_{\v{r}+\hat{x}} + ga(A_{\v{r},y} - A_{\v{r},x})\big)\\
	  &\qquad\qquad+ 2\sin\big(\theta^-_\v{r}-\theta^+_\v{r}\big) + \sin\big(\theta^+_{\v{r}+\hat{x}} - \theta^-_\v{r}+gaA_{\v{r},x}\big) - \sin\big(\theta^-_{\v{r}+\hat{x}} - \theta^+_\v{r} +gaA_{\v{r},x}\big)\\
	&\qquad\qquad-\sin\big(\theta^-_{\v{r}+\hat{y}} - \theta^+_\v{r} + gaA_{\v{r},y}\big) + \sin\big(\theta^+_{\v{r}+\hat{y}} - \theta^-_\v{r} + gaA_{\v{r},y}\big)\Big]\Big\}.
  \end{split}
  \label{eq:disc:freeEn:longDis:MGT}
\end{equation}
The big question is what justification this assumption is based on. In \cite{shimizu12} they also have a two-component $p$-wave order parameter, but they rather assume that
$\sum_h|\eta^h|^2 = \text{const.}$ in what they call the London-approximation. Choosing units such that this constant is $1$, then $\eta^h$ is a complex projective field.

\subsection{London approximation by a $\mathbb{CP}^1$-field}

Instead of fixing the amplitude of each field as in the London approximation, a less strict approximation is to say that
\begin{equation}
  |\eta^+_\v{r}|^2 + |\eta^-_\v{r}|^2 = \gamma^2,
  \label{eq:disc:cp1:squareSum}
\end{equation}
for a constant $\gamma>0$, i.e. the sum of squares of the amplitudes of the components together is constant. This makes it possible that condensate from one component can flow into the other
and vice versa. Defining two new fields
\begin{equation}
  z^h_\v{r} \equiv \frac{1}{\gamma}\eta^h_\v{r} = \frac{1}{\gamma}\rho^h_\v{r}e^{i\theta^h_\v{r}} \equiv u^h_\v{r}e^{i\theta^h_\v{r}},
  \label{eq:disc:cp1:zFields}
\end{equation}
the constraint on $\eta^h_\v{r}$ in Eq.~\eqref{eq:disc:cp1:squareSum} becomes the $\mathbb{CP}^1$ constraint
\begin{equation}
  |z^+_\v{r}|^2 + |z^-_\v{r}|^2 = 1
  \label{eq:disc:cp1:cp1Constraint}
\end{equation}
when written in terms of the new fields. This constraint reduces the real degrees of freedom in the fields from $4$ to $3$. Choosing the amplitude $u^+_\v{r}$ as the free degree of freedom,
the constraint implies that $u^-_\v{r}= \sqrt{1-(u^+_\v{r})^2}$. Rewriting the free energy in Eq.~\eqref{eq:disc:FreeEn:disc} in terms of these new degrees of freedom and ignoring constant terms yields
\begin{equation}
  \begin{split}
	F_K &= \gamma^2\sum_{\v{r},\mu,h}\Big[2|u^h_\v{r}|^2 - \Big(u^h_{\v{r}+\hat{\mu}}u^h_\v{r}e^{-i(\theta^h_{\v{r}+\hat{\mu}}-\theta^h_\v{r} + gaA_{\v{r},\mu})} + \text{c.c.}\Big)\Big]\\
  &= 2\gamma^2\sum_{\v{r},\mu} - 2\gamma^2\sum_{\v{r},\mu,h}u^h_{\v{r}+\hat{\mu}}u^h_\v{r}\cos\big(\theta^h_{\v{r}+\hat{\mu}}-\theta^h_\v{r} + gaA_{\v{r},\mu}\big)\\
	&\sim -2\gamma^2\sum_{\v{r},\mu,h}u^h_{\v{r}+\hat{\mu}}u^h_\v{r}\cos\big(\theta^h_{\v{r}+\hat{\mu}} - \theta^h_\v{r} + gaA_{\v{r},\mu}\big),
  \end{split}
  \label{eq:disc:cp1:FK}
\end{equation}
for the normal kinetic term in Eq.~\eqref{eq:disc:FreeEn:disc:kin},
\begin{equation}
  \begin{split}
	F_V &= a^2\sum_\v{r}\Big\{\sum_h\Big[-\gamma^2u^{h\,2}_\v{r} + \frac{\gamma^4}{2}u^{h\,4}_\v{r}\Big] + 2\gamma^4(u^+_\v{r}u^-_\v{r})^2 + \nu\gamma^4(u^+_\v{r}u^-_\v{r})^2\Re e^{2i(\theta^-_\v{r}-\theta^+_\v{r})}\Big\}\\
	&= a^2\sum_\v{r}\Big\{-\gamma^2 + \frac{1}{2}\gamma^4\big(u^{+\,4}_\v{r} + 1 - 2u^{+\,2}_\v{r} + u^{+\,4}_\v{r}\big) + \gamma^4u^{+\,2}_\v{r}u^{-\,2}_\v{r}\Big[2+\nu\cos 2\big(\theta^+_\v{r}-\theta^-_\v{r}\big)\Big]\Big\}\\
	&= a^2\big(-\gamma^2 + \gamma^4/2\big)\sum_\v{r} + a^2\sum_\v{r}\Big\{\gamma^4u^{+\,2}_\v{r}\big( u^{+\,2}_\v{r}-1\big) + \gamma^4u^{+\,2}_\v{r}u^{-\,2}_\v{r}\Big[2+\nu\cos 2\big(\theta^+_\v{r}-\theta^-_\v{r}\big)\Big]\Big\}\\
	&\sim a^2\gamma^4\sum_\v{r}u^{+\,2}_\v{r}u^{-\,2}_\v{r}\big[1+\nu\cos 2\big(\theta^+_\v{r}-\theta^-_\v{r}\big)\big]
  \end{split}
  \label{eq:disc:cp1:FV}
\end{equation}
for the potential term in Eq.~\eqref{eq:disc:FreeEn:disc:potential} and
\begin{equation}
  \begin{split}
	F_\text{MGT} = \gamma^2\sum_\v{r}\Big\{&(\nu+1)\Big[u^-_\v{r}u^+_{\v{r}+\hat{y}}\cos\big(\theta^+_{\v{r}+\hat{y}} - \theta^-_\v{r} + agA_{\v{r},y}\big)
+ u^+_\v{r}u^-_{\v{r}+\hat{y}}\cos\big(\theta^-_{\v{r} + \hat{y}} - \theta^+_\v{r} + agA_{\v{r},y}\big)\\
&\phantom{(\nu+}\;-\Big( u^-_\v{r}u^+_{\v{r}+\hat{x}}\cos\big(\theta^+_{\v{r}+\hat{x}} - \theta^-_\v{r} + agA_{\v{r},x}\big)
+ u^+_\v{r}u^-_{\v{r}+\hat{x}}\cos\big(\theta^-_{\v{r} + \hat{x}} - \theta^+_\v{r} + agA_{\v{r},x}\big)\Big)\Big]\\
+&(\nu-1)\Big[u^+_{\v{r}+\hat{y}}u^-_{\v{r}+\hat{x}}\sin\big(\theta^-_{\v{r}+\hat{x}} - \theta^+_{\v{r}+\hat{y}} + ga(A_{\v{r},x} - A_{\v{r},y})\big)\\
  &\phantom{(\nu+}\;\,-u^-_{\v{r}+\hat{y}}u^+_{\v{r}+\hat{x}}\sin\big(\theta^+_{\v{r}+\hat{x}} - \theta^-_{\v{r}+\hat{y}} + ga(A_{\v{r},x} - A_{\v{r},y})\big) + 2u^+_\v{r}u^-_\v{r}\sin\big(\theta^-_\v{r}-\theta^+_\v{r}\big)\\
  &\phantom{(\nu0}\;\,+u^-_\v{r}u^+_{\v{r}+\hat{x}}\sin\big(\theta^+_{\v{r}+\hat{x}} - \theta^-_\v{r}+gaA_{\v{r},x}\big)
  -u^+_\v{r}u^-_{\v{r}+\hat{x}}\sin\big(\theta^-_{\v{r}+\hat{x}} - \theta^+_\v{r}+gaA_{\v{r},x}\big)\\
  &\phantom{(\nu0}+u^-_\v{r}u^+_{\v{r}+\hat{y}}\sin\big(\theta^+_{\v{r}+\hat{y}} - \theta^-_\v{r}+gaA_{\v{r},y}\big)
-u^+_\v{r}u^-_{\v{r}+\hat{y}}\sin\big(\theta^-_{\v{r}+\hat{y}} - \theta^+_\v{r}+gaA_{\v{r},y}\big)\Big]\Big\},
  \end{split}
  \label{eq:disc:cp1:MGT}
\end{equation}
for the mixed gradient terms in Eq.~\eqref{eq:disc:FreeEn:disc:MGT}. The Maxwell term is unchanged from its form in Eq.~\eqref{eq:disc:Maxwell:plaquetteSum}.

For the numerics we set $a=1$ and rescale the gauge field s.t. $A_{\v{r},\mu} \mapsto -\frac{1}{ga}A_{\v{r},\mu}$. Then the free energy is $F = F_K + F_V + F_\text{AB} + F_\text{MGT} + F_A$ where
the different energies take the form
\begin{align}
  &F_K = -2\gamma^2\sum_{\v{r},\mu,h}u_{\v{r}+\hat{\mu}}^hu_\v{r}^h\cos\big(\theta_{\v{r}+\hat{\mu}}^h-\theta_\v{r}^h - A_{\v{r},\mu}\big),\label{eq:disc:cp1:rescale:FK}\\
  &F_V = \gamma^4\sum_\v{r}\big(u_\v{r}^+u_\v{r}^-\big)^2\Big[1+\nu\cos2\big(\theta_\v{r}^+ - \theta_\v{r}^-\big)\Big], \label{eq:disc:cp1:rescale:FV}\\
  &F_A = \frac{1}{g^2}\sum_\v{r}\big(A_{\Box,\v{r}}^f\big)^2, \label{eq:disc:cp1:rescale:FA}\\
  \begin{split}
	F_\text{AB} = \gamma^2(\nu+1)\sum_\v{r}\Big\{&\Big[u_\v{r}^-u_{\v{r}+\hat{y}}^+\cos\big(\theta_{\v{r}+\hat{y}}^+ - \theta_\v{r}^- - A_{\v{r},y}\big) 
  - u_\v{r}^-u_{\v{r}+\hat{x}}^+\cos\big(\theta_{\v{r}+\hat{x}}^+-\theta_\v{r}^--A_{\v{r},x}\big)\Big]\\ + &\Big[+\leftrightarrow-\Big]\Big\},
\end{split}\label{eq:disc:cp1:rescale:FAB}\\
  \begin{split}
	F_\text{MGT} = \gamma^2(\nu-1)\sum_\v{r}&\Big\{\Big[u_{\v{r}+\hat{y}}^+u_{\v{r}+\hat{x}}^-\sin\big(\theta_{\v{r}+\hat{x}}^--\theta_{\v{r}+\hat{y}}^+ - (A_{\v{r},x}-A_{\v{r},y})\big)\\
	&+ u_\v{r}^-u_{\v{r}+\hat{x}}^+\sin\big(\theta_{\v{r}+\hat{x}}^+-\theta_\v{r}^--A_{\v{r},x}\big) + u_\v{r}^-u_{\v{r}+\hat{y}}^+\sin\big(\theta_{\v{r}+\hat{y}}^+-\theta_\v{r}^--A_{\v{r},y}\big)\Big]\\
	- &\Big[+\leftrightarrow-\Big] + 2u_\v{r}^+u_\v{r}^-\sin\big(\theta_\v{r}^--\theta_\v{r}^+\big)\Big\}.
  \end{split}\label{eq:disc:cp1:rescale:FMGT}
\end{align}
Here we have split the previous $F_\text{MGT}$ energy into the Andreev-Bashkin terms $F_\text{AB}$, and the true mixed gradient terms $F_\text{MGT}$.
