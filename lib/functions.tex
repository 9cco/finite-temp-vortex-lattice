\usepackage{amsmath}
\usepackage{amssymb}
\usepackage{amsthm}	% For theorems.
\usepackage{siunitx}	% For SI units.
\usepackage{slashed}	% For Feynman Slash notation
\usepackage{mathrsfs}	% For fancy script font.
\usepackage{tikz}
\usetikzlibrary{arrows,shapes,trees,decorations.markings, calc,tikzmark}
\usepackage[compat=1.1.0]{tikz-feynman}
\usepackage{mathtools}
% Configuring space above and below theorems
\makeatletter
\def\thm@space@setup{%
  \thm@preskip=0.5in plus 0.1in minus 0.2in
  \thm@postskip=\topsep % or whatever, if you don't want them to be equal
}
\makeatother

%\renewcommand{\theenumi}{(\alph{enumi})}
%\renewcommand{\labelenumi}{\theenumi} % Labeling in lists.
% For slanted fractions.
\newcommand*\sfrac[2]{{}^{#1}\!/_{#2}}
% For derivatives.
\newcommand{\pd}[2]{\frac{\partial #1}{\partial #2}} 
\newcommand{\pdd}[2]{\frac{\partial^2 #1}{\partial #2^2}}
\newcommand{\pddt}[3]{\frac{\partial^2 #1}{\partial #2\,\partial #3}}
\newcommand{\fd}[1]{\frac{\delta}{\delta #1}}	% Functional derivative
\let\underdot=\d % rename builtin command \d{} to \underdot{}
\renewcommand{\d}[2]{\frac{\mathrm{d} #1}{\mathrm{d} #2}} % for derivatives
\let\vaccent=\v % rename builtin command \v{} to \vaccent{}
\renewcommand{\v}[1]{{\ensuremath{\boldsymbol{\mathbf{#1}}}}} % for vectors
\newcommand{\vv}[1]{\ensuremath{\overrightarrow{\boldsymbol{\mathbf{#1}}}}} % for vectors with arrow
\newcommand{\gv}[1]{\ensuremath{\mbox{\boldmath$ #1 $}}} 
% for vectors of Greek letters
\newcommand{\uv}[1]{\ensuremath{\mathbf{\hat{#1}}}} % for unit vector
\newcommand{\norm}[1]{{\left\|#1\right\|}}
\newcommand{\ket}[1]{\left| #1 \right>} % for Dirac kets 
\newcommand{\ketp}[1]{\left| #1 \right)} % for kets with )
\newcommand{\ketb}[1]{\left| #1 \right\}} % for kets with }
\newcommand{\bra}[1]{\left< #1 \right|} % for Dirac bras
\newcommand{\brap}[1]{\left( #1 \right|}
\newcommand{\brab}[1]{\left\{ #1 \right|}
\newcommand{\braket}[2]{\left< #1 \vphantom{#2} \right|
 \left. \!\!#2 \vphantom{#1} \right>} % for Dirac brackets
\newcommand{\brakep}[2]{\left( #1 \vphantom{#2} \right|
 \left. \!\!#2 \vphantom{#1} \right)} % for Dirac brackets with ()
\newcommand{\brakeb}[2]{\left\{ #1 \vphantom{#2} \right|
\left. \!\!#2 \vphantom{#1} \right\}} % for Dirac brackets with {}

\newcommand{\avg}[1]{\left< #1 \right>} % for average
\newcommand{\opavg}[1]{\big< #1 \big>} % not so large average
\newcommand{\abs}[1]{\left| #1 \right|} % for absolute value
\newcommand{\dd}[2]{\frac{d^2 #1}{d #2^2}} % for double derivatives
\newcommand{\ip}[2]{{\langle #1,#2\rangle}}  % for inner products
\newcommand{\ten}[3]{#1_{#2}^{\phantom{#2}#3}} % for down-up tensors.
\DeclareMathOperator{\Tr}{Tr} % Trace operator
\DeclareMathOperator{\tr}{tr}
\renewcommand{\Re}{\operatorname{Re}}
\renewcommand{\Im}{\operatorname{Im}}
\newcommand{\Ham}{\mathscr{H}}	% Hamiltonian operator
\newcommand{\Hamop}{\hat{\mathcal{H}}} 	% Hamiltonian operator
\newcommand{\Id}{\mathbb{I}}	% Identity operator
\newcommand{\alphav}{\gv{\alpha}}	% Dirac alpha vector
\newcommand{\pauliv}{\gv{\sigma}}	% Pauli matrix vector
\newcommand{\piv}{\gv{\pi}}		% Effective momentum in minimal coupling
\newcommand{\inte}[1]{\int\!\!\mathrm{d}#1\,}	% Integral with measure in front
\newcommand{\dint}[2]{\int\!\!\mathrm{d}#1\mathrm{d}#2\,}	% Double integral with measure in front.
\newcommand{\up}{\uparrow}
\newcommand{\down}{\downarrow}
\newcommand{\pathint}[1]{\int\!\mathcal{D}[#1]\,}	% Path integral
\newcommand{\pathintlim}[2]{\int\limits_{\mathrlap{#1}}\!\mathcal{D}[#2]\,} % For providing a lower limit in a path-integral that extends towards the right. Btw there are also functions mathclap,
																		% which centers the content, and mathllap which right justifies it
\newcommand{\ferm}[1]{n_\text{F}(#1)}				% Shorthand for the symbol of the fermi-distribution-function.
\newcommand{\diag}[1]{\mathrm{diag}\left(#1\right)}
\newcommand{\rep}[1]{\hat{D}(#1)}					% Operator represention of group element #1
\newcommand{\bos}[1]{\boldsymbol{#1}}
\newcommand*\circled{\:\tikz[baseline=(char.base)]{
            (0,0) \node[shape=circle,draw,inner sep=-3.4pt](char){$\sum$};}\:}
\newcommand{\rom}[1]{%
  \textup{\uppercase\expandafter{\romannumeral#1}}  % For roman numerals
}
\newcommand*\chem[1]{\ensuremath{\mathrm{#1}}}

% AMS-LaTeX
%\usepackage{amsmath}
%\usepackage{amsfonts}
%\usepackage{amssymb}

\DeclareMathOperator{\clos}{clos}
\DeclareMathOperator{\sgn}{sgn}
\DeclareMathOperator{\Span}{span}
\DeclareMathOperator{\ran}{ran}
\DeclareMathOperator*{\Res}{Res}
\DeclareMathOperator{\Pf}{Pf}

% Set various options, like unnumbered sections
%\setcounter{secnumdepth}{0}

%\pagestyle{empty}	% Turn of page numbers.




% Theorems
\theoremstyle{plain}
\newtheorem{thm}{Theorem}[section]
\newtheorem{corr}[thm]{Corollary}
\newtheorem{prop}[thm]{Proposition}
\newtheorem{lem}[thm]{Lemma}
\theoremstyle{definition}
\newtheorem{dfn}{Definition}
\newtheorem{ex}[dfn]{Example}
\theoremstyle{remark}
\newtheorem*{rmk}{Remark}

% Use a filled qed symbol for the end of proofs
\renewcommand{\qedsymbol}{$\blacksquare$}
